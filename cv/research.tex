\documentclass{amsart}

\usepackage{fullpage}
\usepackage{amsmath}
\usepackage{amssymb}
\usepackage{graphicx}
\usepackage[sort&compress]{natbib}
\usepackage[
  letterpaper=true,
  colorlinks=true,
  linkcolor=black]{hyperref}

\usepackage{fancyhdr}

\title{Research statement}
\author{David Bindel}

\newcommand{\bbC}{\mathbb{C}}
\newcommand{\bbR}{\mathbb{R}}

\begin{document}

\maketitle
\vspace{-5mm}
\begin{center}
\url{http://www.cs.cornell.edu/~bindel/}
\end{center}
\vspace{+5mm}

%\section*{Introduction}

In my research in scientific computing,
I develop models, approximations, algorithms, and software that
are fast and accurate because they use problem structure --- the
``physics'' of the application, whether or not the problem is physical.
%
My work is collaborative and interdisciplinary: to ask sensible questions
requires application expertise; to design and analyze solution
algorithms requires mathematical expertise; and to write efficient
codes requires software expertise.
%
I work with colleagues across many application areas, partly from
broad intellectual curiosity, and partly to explore the mathematical
structure these problems bring:
%
structure from the geometry of a physical problem,
special algebraic properties like complex symmetry or
low-rank structure, and
analytic properties involving perturbations from some geometric or
algebraic structure.

Most of my work involves applications of linear algebra, and
particularly spectral analysis methods.
%
Eigenvalue problems have many applications, largely because they among
the few nonlinear equations for which we have global solution methods;
and symmetric eigenvalue problems can be formulated as one of the few
non-convex optimization problems for which we have good general-purpose
methods.
%
%have used structured eigenvalue problems to design musical
%instruments~\cite{2006-sound,2007-sound}, to design resonant
%micro-mechanical systems~\cite{2013-sensors,2016-sensors,2005-ijnme}, to
%analyze bifurcations in dynamical systems~\cite{2014-matcont,2008-cis},
%and for a variety of other applications.
%
In Section~\ref{sec-nep}, I describe my work on the theory and
applications of {\em nonlinear eigenvalue problems}.
%
In Section~\ref{sec-eigen-apps}, I describe spectral analysis applied to
problems from different areas of computer science; and in
Section~\ref{sec-sc-systems}, I describe my work specifically on systems
and scientific computing. In Section~\ref{sec-engineering}, I describe
an engineering application, the design of simulation tools for
micro-machined resonant micro-gyroscopes. I conclude in
Section~\ref{sec-directions} by discussing other directions
that I am actively pursuing with students.

% Pointer to systems stuff
% Opt

% LA + graph theory, geometry, complex and functional analysis,
% group theory, continuum mechanics.

\section{Nonlinear eigenvalue problems and applications}
\label{sec-nep}

\subsection*{Nonlinear eigenvalue problems}
Given an analytic matrix-valued function
$T : \Omega \subset \bbC \rightarrow \bbC^{n \times n}$,
the nonlinear eigenvalue problem is to find an eigenvalue
$\lambda \in \bbC$ such that $T(\lambda)$ is singular.
The standard eigenvalue problem for a matrix $A$
is a special case of the nonlinear eigenvalue problem where
$T(z) = A-zI$.  Nonlinear eigenvalue problems often arise
when transform methods are used to analyze differential and
difference equations; for a survey of applications, we refer to~\cite{2015-sirev}.

For the standard eigenvalue problem, there
are many eigenvalue localization results that can be used for error
analysis or to answer general qualitative questions, such as whether a
given matrix might have any eigenvalues in the right half-plane.  There
are multiple proofs for most of these results, but they all admit proofs
based on analytic function theory.  These proofs do not explicitly take
advantage of linearity of the function $A-\lambda I$ in $\lambda$, and
consequently extend to nonlinear eigenvalue problems as well, with
appropriate modifications.  Together with Amanda Hood, an applied math
Ph.D. student, I developed a general framework for these types of
eigenvalue problems and gave several examples~\cite{2013-simax}.  This
work was awarded the SIAG/LA best paper award, and appeared as a SIGEST
feature article in {\em SIAM Review}~\cite{2015-sirev}.

In our ongoing work, we use nonlinear pseudospectra to bound the
transient dynamics of linear delay-differential equations in the same
way that one can use ordinary pseudospectra to bound the transient
dynamics of ordinary linear differential equations.  We are also working
to use our results to obtain rigorous bounds on resonance poles from
multi-dimensional scattering problems.

\subsection*{Applications to resonances}

My interest in nonlinear eigenvalue problems grew from my work on
modeling anchor loss in high-frequency micro-resonators; that is,
damping due to radiation of waves from the resonating structure into a
(effectively semi-infinite)
substrate~\cite{2004-para,2005-mems,2005-ijnme,2005-sensors}.  This is a
problem of computing {\em resonance poles}, which give information about
the dynamics of trapped waves leaking from a bounded
subdomain in the same way that eigenvalues give information about
steady-state vibrations on finite domains. One method to
compute resonances is to truncates the semi-infinite domain with a
(possibly approximate) radiating boundary condition, which depends very
nonlinearly on the frequency.

From this starting point, I became more interested in resonance poles in
general, which lead to a fruitful collaboration with Maciej Zworski at
Berkeley involving exploring resonances in a simplified
setting~\cite{2007-symmetry}.  As part of my collaboration with Zworski,
I developed a code ({\tt MATSCAT}) to compute resonances for
Schr{\"o}dinger problems in one space dimension with compactly-supported
potentials in one space dimension.  For this problem, I was able to
develop a fairly complete analysis for the discretized problem as well
as a first-order backward error analysis to estimate the impact of
discretization error.  This technique does not generalize to higher
dimensions, nor the case of an inhomogeneous far-field behavior.  But
my work with Amanda Hood on nonlinear eigenvalue problem design provides
the basis for a more general error analysis.

\subsection*{Applications to evolutionary game theory}

Inspired by the thesis work of Danielle Toupo, a recent applied math
Ph.D. graduate, I became intrigued by problems in
evolutionary game theory where species interact through competition for
a common resource.  Together with Steve Strogatz and Irena Papst,
another applied math Ph.D.~student, I have begun to explore a
special case involving systems of the form
\[
  \dot{x} = A(r) x, \quad  0 = g(\dot{r}, r, x)
\]
where $r(t) \in \bbR$ represents the amount of a shared resource at
time $t$, $x(t) \in \bbR^n$ represents the population level for $n$
different species or strategies, and $A(r) \in \bbR^{n \times n}$
represents growth and mutation terms.  The equilibrium equation is
the nonlinear eigenvalue problem $A(r) x = 0$.

\section{Spectral analysis in Computer Science}
\label{sec-eigen-apps}

In this section, I describe applications of spectral analysis to
different areas of computer science: data mining, computer vision,
machine learning, game theory, and network analysis. In each case, my
contributions center on finding and exploiting structure.

% For each problem:
% - Overview and how I got into it
% - Key structural features + mathematical connections
% - Main results so far
% - Current / future directions (if any)

\subsection*{Model reduction and PageRank}

PageRank is a standard tool for finding important vertices in a graph,
based on a ``random surfer'' model that converges to
a stationary distribution vector $x$ satisfying
\[
  (I-\alpha P) x = (1-\alpha) x^*,
\]
where $P$ is a column-stochastic transition matrix representing a random
walk on the graph, and the parameter $0 \leq \alpha < 1$  represents the
probability of taking a step according to $P$ rather than
``teleporting'' to a new node drawn from the reference distribution
$x^*$.  PageRank can be {\em personalized} to find vertices
most relevant to a query or user; in the most common form of
personalization, we use a
reference distribution $x^*(\omega)$ that depends on
a personalization parameter vector $\omega$.  In joint work with Wenlei
Xie, Johannes Gehrke, and Al Demers, we consider instead the problem
with a personalized {\em transition matrix} $P(\omega)$;
for example, if there are a few different types of edges,
then $\omega$ might control how these edge types are weighted.

Our paper~\cite{2015-edgeppr}, which won a best student paper award at KDD
2015, describes the first fast algorithm for computing PageRank on
general graphs when the edge weights are personalized. Our method is
based on model reduction, a standard technique for physical simulations
that I used in my previous work on
MEMS~\cite{2001-msm,2004-para,2005-ijnme,2005-sensors}.  The
key observation is that if the parameters vary over a compact domain and
$P(\omega)$ has bounded derivatives, then by standard interpolation
results, $x(\omega)$ is well approximated by an interpolant through a
few sample points, and therefore must lie near a low-dimensional
subspace.  Our method learns this subspace in an offline phase, and
enforces a few equations in the online phase in
order to reconstruct a good approximation to $x(\omega)$.  Our approach
outperforms prior methods by nearly five orders of magnitude, a
performance gain that enables new applications, such as
solving learning-to-rank problems for edge weight personalization at
interactive speeds.

\subsection*{Eigenvalues and hardness of rotation averaging}

One key subproblem in reconstructing three-dimensional models from
photograph collections is {\em rotation averaging}: given estimates of
the relative rotations $\tilde{R}_{ij}$ between the cameras for many
pairs of images $(i,j)$, how can we assign camera rotations $R_i$ to
each image $i$ in a way that is most consistent with the data?  We
consider a least squares formulation
\[
  \min_{(R_i)_{i=1}^n} \sum_{(i,j)
  \in \mathcal{E}} d(\tilde{R}_{ij}, R_i R_j^T)^2
\]
where $d$ measures the geodesic distance between two elements of
the rotation group $SO(3)$.  While there exist good algorithms for
rotation averaging, these methods sometimes fail.  In work with
Noah Snavely and Kyle Wilson, an applied math Ph.D. student,
we explored an approach to characterizing which rotation averaging
problems are easy and which are hard.
%
I contributed the observation that while the objective is nowhere convex
on $SO(3)^n$, it is constant over the orbits generated by action of
$SO(3)$; and local convexity is indeed possible if one works on the
quotient manifold $SO(3)^n/SO(3)$. From there, we were able together to
relate local convexity in the neighborhood of a solution to the smallest
eigenvalue of a normalized graph Laplacian with boundary conditions in
which the weight of edge $(i,j)$ is determined by the magnitude of the
residual errors $d(\tilde{R}_{ij}, R_i R_j^T)$.  This work will appear
in ECCV 2016~\cite{2016-rotations}, and Kyle Wilson and I continue to
collaborate on further extensions.

% Connection to geometry

\subsection*{Spectral topic modeling}

Topic modeling is a statistical approach to understanding the
structure of a document collection.  In most topic models,
we express each document as a distribution over a few topics,
and each topic as a distribution over words.
Standard generative topic models, such as Latent Dirichlet Allocation,
provide a high-quality decomposition, but are expensive to train.
In contrast, recent spectral inference algorithms based on
a separability assumption (the ``anchor
word'' assumption) are both fast and provably optimal under certain
assumptions on the generating process. We consider spectral methods
based on a matrix of co-occurrence statistics that
describe the probability with which terms appear together in a document.
In particular, we define a generative model for the matrix of
co-occurrence statistics, the {\em Joint Stochastic Matrix
Factorization} (JSMF) model.  For a vocabulary of size $n$ and $k$
topics, the JSMF model predicts that (in expectation) the
co-occurrence matrix $C\in \bbR^{n \times n}$ admits the
factorization
\[
  C \approx B A B^T
\]
where $B \in \bbR^{n \times k}$ is
a column-stochastic matrix in which each column represents the word
distributions for one topic, and $A \in \bbR^{k \times k}$ is a
positive-definite jointly stochastic matrix whose $(i,j)$ entry
represents the probability that topics $i$ and $j$ occur together.

According to the JSMF model, the matrix of co-occurrence statistics $C$
should be jointly stochastic (i.e.~the entries should be non-negative
and should sum to one) and it should be symmetric positive definite with
rank $k$.  Real data sets are often inconsistent with these assumptions.
In~\cite{2015-nips}, we show how to enforce this structure in
sample co-occurrence statistics by an alternating projection algorithm,
and show that training on the resulting pre-processed data significantly
improves the quality of inferred topics.

\subsection*{Eigenvalues in algorithmic game theory}

In economics and sociology, opinions held by individuals are often
modeled as numerical values, and people arrive at a shared opinion by
repeatedly averaging their opinion with the opinions of neighbors in a
social network.  In a connected network, pure repeated averaging leads
to an equilibrium with constant opinions; but in
real life, consensus is rarely reached. In joint work with Jon
Kleinberg and Sigal Oren (then a CS Ph.D.~student),
we study a related model in which individuals’ intrinsic beliefs
counterbalance the averaging process and yield a diversity of opinions~\cite{2011-focs,2015-geb}.
By interpreting the repeated averaging as best-response dynamics in an
underlying game with natural payoffs, and the limit of the process as an
equilibrium, we are able to study the cost of disagreement in these
models relative to a social optimum.

My contribution to this work involves the insight that the costs
for this game at both the Nash equilibrium and the social optimum are
quadratic forms in the vector of intrinsic beliefs.
Because minimizing a ratio of quadratic forms leads to an eigenvalue
problem, one can reason about tight bounds on the price of anarchy ---
the ratio between the cost at Nash equilbrium and the social optimum ---
for all possible values of the intrinsic beliefs by bounding extremal
eigenvalues of a generalized eigenvalue problem.
Both matrices in this problem can be expressed in terms of a
weighted graph Laplacian.  In the undirected case, the two matrices are
simple polynomial functions of the Laplacian, and so the price of
anarchy is bounded by the minimum of a rational function over the
spectrum of the Laplacian (which is in turn bounded by the minimum of
this function over the positive numbers).  The directed case is more subtle.

We also consider a natural network design problem in this setting: which
links can we add to the underlying network to reduce the cost of
disagreement at equilibrium? In this part of the work, my contribution
was to show that we can optimally add a small amount of edge {\em weight}
by first-order sensitivity analysis of the cost function, even though
optimally adding an edge is NP-hard.

\subsection*{Local spectral methods for clustering}

Spectral clustering methods partition networks into large
clusters that approximately minimize the number of edges between
clusters or maximize quality measures such as modularity. But these
methods do not find small communities, unless we use them
recursively.
In contrast, diffusion methods explore the neighborhood around a
pre-defined set of ``seed'' nodes in a network in order to find small
communities that are good in some sense (using a measure like
conductance).  Our work~\cite{2016-losp-kdd,2015-icdm,2015-www}
interpolates between spectral methods and diffusion-based
methods by searching for communities using unconverged eigenvector
estimates based on a few steps from standard eigensolver iteration
started from a vector supported on the seed set. These eigenvector-like
objects reflect the shape of the intermediate dynamics of diffusion
processes in a manner analogous to how the extremal eigenvectors of a
transition matrix reflect the shape of the asymptotic dynamics of
convergence to the equilibrium distribution.  Using minimal
$\ell_1$-norm linear combinations of unconverged eigenvectors
to approximate indicators for clusters, we are able to find
clusters in real-world networks that are closer to the ground-truth
labels than clusters from other standard methods such
as heat kernel diffusion and PageRank diffusion.

\section{Scientific computing for systems}
\label{sec-sc-systems}

My collaboration with systems researchers has involved two distinct
themes. First, we work on frameworks that simplify the process of
writing high performance code.  For example, in a collaboration with the
Cornell database group, we extended the GRACE graph processing engine
with blocking ideas common in the design of high-performance iterative
methods~\cite{2013-blockgrace}.  By processing nodes that are ``close''
in a graph together, we both reduce the cost per iteration (because of
cache locality effects) and reduce the number of iterations required to
obtain convergence.  In~\cite{2011-socc}, we designed a programming
framework and runtime for parallel bulk-synchronous iterations on cloud
platforms.  The message latency in these systems often has high
variance, and our system finds work to do while waiting for high-latency
messages; as a result, we speed up typical computations by a factor of
three compared to a more conventional bulk synchronous implementation.
More recently, I have written platform to support asynchronous parallel
optimization codes ({\tt POAP}), which supports a surrogate-based global
optimization library ({\tt PySOT}).

The second theme in my systems collaboration is to build and analyze
numerical models of systems.  For example, I recently collaborated with Gun Sirer, Robert Van Renesse, and CS Ph.D.~student Efe Gencer on
work to use response surfaces to optimize software and hardware
configurations for distributed computations~\cite{2015-middleware}.  Our
approach yielded improvements of up to $5 \times$ compared to the
default configurations, and our paper won a best paper prize at
Middleware 2015.  Earlier, I collaborated with Yan Chen on several papers
on using linear algebra to find link failures in computer networks based
on end-to-end path
measurements~\cite{2009-tons,2007-tons,2006-sigcomm,2006-sigmetrics,2004-sigcomm,2003-imc}.

\section{Micro-electro-mechanical systems}
\label{sec-engineering}

I have worked extensively on computer-aided design (CAD) tools for
Micro-Electro-Mechanical Systems (MEMS), micrometer-scale devices used
in everything from digital projectors to car engines.  In the past, I
worked on system-level simulation software for MEMS
({\tt SUGAR})~\cite{2002-mems,2001-sugar,2001-msm,2000-mems}
and on finite element simulations of damping in high-frequency
MEMS resonators
({\tt HiQLab})~\cite{2012-mems-opt,2005-sensors,2005-ijnme,2005-mems,2004-para}.
More recently, I have collaborated with Sunil Bhave and
Erdal Yilmaz, an applied physics Ph.D.~student, on micro-scale
solid-wave gyroscopes.

Solid-wave hemispherical resonator gyroscopes (HRGs) have become the
gyroscope of choice for satellites and other spacecraft.  Like Foucault's
pendulum, these devices measure total rotation through the transfer
of energy between two geometrically degenerate modes of vibration.
Prompted by questions from our collaborators about why different designs
seemed to work better with degenerate mode pairs associated with
different azimuthal wave numbers, we discovered how symmetry-breaking
geometric perturbations associated with different micro-fabrication
processing steps can break the degeneracy of some mode pairs and
not others.
In~\cite{2013-sensors}, we describe how to use
{\em selection rules} based on the representation theory of
finite groups to reason about the effect of different microfabrication
artifacts on device performance; our approach is the
same one used in chemistry and physics when reasoning about the
effect of external fields on the electronic structure of
symmetric atoms and molecules.  We also describe a finite
element code ({\tt AxFEM}) that uses a ``2.5D'' formulation to
produce fast, accurate analysis of the influence of symmetry-breaking
perturbations to the geometry of the nominally axisymmetric
device design.  In~\cite{2016-sensors,2016-hh-workshop}, we use
the same methods to analyze the temperature sensitivity of these
micro-gyros.

\section{Further directions}
\label{sec-directions}

Beyond the work mentioned in the previous sections, there are
several other earlier-stage projects that I am pursuing with my
graduate students.

\subsection*{Graph Density of States}

% Make the spectral geometry connection

Researchers from many disciplines study spectra to understand the
structure and composition of mathematical objects and physical systems.
For large systems, {\em complete} spectral information costs too much,
whether it is gathered computationally or measured experimentally;
hence, most spectral analysis methods use {\em partial} information
about the eigenvalues or eigenvectors of some matrix or operator.
These methods use a few (extreme) eigenvalues and
associated eigenvectors; invariants that are simple functions of all the
eigenvalues; or densities of eigenvalues (a.k.a.~the {\em density of states}),
possibly weighted by local importance (the {\em local density of states}).

All three approaches are used in spectral geometry an in applications to
physics and engineering systems.  But distributional information is much
less used in the study of complex networks, even though the local
density of states at each node is strictly more informative than
some standard centrality measures such as subgraph centrality.
Together with Kun Dong, an applied math Ph.D.~student, and a team of
undergraduate researchers, I am currently developing a MATLAB package
({\tt GraphDoS}) to compute global and local densities of states.
Our approach uses standard density estimation techniques used in
computational material science, but also uses techniques that are
specialized to network analysis~\cite{2015-siam-ns}.  Our ultimate
goal is to be able to be able to quickly compute, visualize, and
interpret graph densities of states in the same
way that one interprets the density of states for a material system.

\subsection*{Electrical power grids}

% Observe connection to network tomography
%\cite{2016-flier-tr}

The power grid is changing, driven by many factors: the push to
increased renewable energy sources, highly dynamic power markets, and
deployment high-resolution measurements of transmission networks by
syncrophasor measurement units (PMUs).  In work with Colin Ponce,
my former CS Ph.D.~student, we consider the problem of fusing information
from old and slow SCADA sensor systems, which provide complete
observability and from which an estimate of system state is
reconstructed every few minutes, with the much faster (but less
complete) PMU information.  Our system, FLiER~\cite{2016-flier-tr},
uses the difference between PMU measurements and the
behavior predicted by state estimation from SCADA data to infer
the likelihood of a variety of topology changes in the network,
including line outages, substation reconfigurations, generator
trips, and load trips.  Because the system state changes over
time, the ``fingerprints'' of these different changes cannot be
pre-computed; therefore, FLiER uses a novel filtering mechanism
that rules out most possible events without
the cost of even a partial simulation of the event.

FLiER is formulated in terms of the quasi-static behavior of the
power grid after a topology change, and filters out the potentially
informative ringdown dynamics that can be seen in the PMU data.  In
current work with my CS Ph.D.~student Eric Lee, we also match observed
ringdown dynamics to the dynamic mode shapes predicted under different
contingencies in order to improve the diagnostic accuracy of the FLiER
approach.

\subsection*{Surrogate optimization methods}

Surrogate methods (also called response surface methods) fit a model
function (the surrogate) to the objective function based on data
sampling.  The surrogate model is then used to guide where the algorithm
should sample next.  Together with Christine Shoemaker and David
Erikssson, an applied math Ph.D.~student, I am pursuing new surrogate
optimization methods that support asynchronous parallel execution.  This
is of particular interest when it is possible to get potentially
incomplete function information (e.g. upper or lower bounds) at much
less expense than a full function evaluation. Our Python Surrogate
Optimization Toolbox ({\tt PySOT}) implements several parallel surrogate
optimization methods on a variety of computing platforms, using our
package of Python Optimization Asynchronous Plumbing ({\tt POAP}) to
handle the details of asking workers for function evaluations.

\newpage
\bibliographystyle{plain}
\bibliography{references}

\end{document}
